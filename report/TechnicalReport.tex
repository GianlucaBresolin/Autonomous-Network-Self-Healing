\documentclass[conference]{IEEEtran}
\IEEEoverridecommandlockouts

\usepackage{cite}
\usepackage{amsmath,amssymb,amsfonts}
\usepackage{algorithmic}
\usepackage{graphicx}
\usepackage{textcomp}
\usepackage{xcolor}
\def\BibTeX{{\rm B\kern-.05em{\sc i\kern-.025em b}\kern-.08em
    T\kern-.1667em\lower.7ex\hbox{E}\kern-.125emX}}
\begin{document}

\title{Autonomous Network Self-Healing in Drone Swarms}

\author{\IEEEauthorblockN{1\textsuperscript{st} Riccardo Fabbian}
\IEEEauthorblockA{\textit{University of Padua: Computer Science} \\
\textit{University of Padua}\\
Padua, Italy \\
fabb.riccardo@gmail.com}
}

\author{\IEEEauthorblockN{2\textsuperscript{nd} Gianluca Bresolin}
\IEEEauthorblockA{\textit{University of Padua: Computer Science} \\
\textit{University of Padua}\\
Padua, Italy \\
gianbreso02@gmail.com}
}

\maketitle

\begin{abstract}
This research reports the formation control method for a swarm of drones
capable, after the loss of communication between a drone and the base station, 
of autonomously reorganizing to form a multi-hop relay chain. \\
The proposed method is based on a virtual spring dumper model, where drones
are attracted by preceding and following drones in the multi-hop relay chain, 
while being repelled by other drones in the swarm to avoid collisions. \\
The experiment was conducted in \textit{ns-3}, a network simulator, where we 
modeled a swarm of drones communicating through radio signals and moving in a 3D
space. \\
Results show that the proposed method is feasible and allows the swarm to
successfully reorganize itself after the loss of communication of a drone,
forming a stable multi-hop relay chain with intermediary drones positioned 
close to the geometric midpoint between preceding and following ones.
\end{abstract}

\begin{IEEEkeywords}
Drones, Swarm, Wireless, Flooding, Self-Healing.
\end{IEEEkeywords}

\section{Introduction}
\section{Self-Healing Formation Control}
\section{Simulation Tested}
\section{Results and Analysis}
\section{Conclusions}

\section*{Acknowledgments}

\section*{References}
% Please number citations consecutively within brackets \cite{b1}. 

% \begin{thebibliography}{00}
% \bibitem{b1} G. Eason, B. Noble, and I. N. Sneddon, ``On certain integrals of Lipschitz-Hankel type involving products of Bessel functions,'' Phil. Trans. Roy. Soc. London, vol. A247, pp. 529--551, April 1955.
% \bibitem{b8} D. P. Kingma and M. Welling, ``Auto-encoding variational Bayes,'' 2013, arXiv:1312.6114. [Online]. Available: https://arxiv.org/abs/1312.6114
% \bibitem{b9} S. Liu, ``Wi-Fi Energy Detection Testbed (12MTC),'' 2023, gitHub repository. [Online]. Available: https://github.com/liustone99/Wi-Fi-Energy-Detection-Testbed-12MTC
% \bibitem{b10} ``Treatment episode data set: discharges (TEDS-D): concatenated, 2006 to 2009.'' U.S. Department of Health and Human Services, Substance Abuse and Mental Health Services Administration, Office of Applied Studies, August, 2013, DOI:10.3886/ICPSR30122.v2
% \end{thebibliography}

\end{document}
